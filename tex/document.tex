\documentclass[a4paper,pagesize=pdftex,12pt]{article}

\usepackage[utf8]{inputenc}
\usepackage[french]{babel}
\usepackage[T1]{fontenc}
\usepackage{lmodern}
\usepackage[protrusion=true,final]{microtype}

\usepackage[pdftex]{hyperref}
\makeatletter
\AtBeginDocument{
  \hypersetup{
    pdftitle = {\@title},
    pdfauthor = {\@author}
  }
}
\makeatother

\author{Jim McCarthy, Michele McCarthy}
\title{Les Protocoles Fondamentaux v3.03}
\date{}

% Headers and footers
\usepackage{fancyhdr}
\pagestyle{fancy}
\rhead{McCarthy Technologies, Inc.}
\lfoot{Version 3.03}
\cfoot{}
\rfoot{\thepage}

\begin{document}

\maketitle

\noindent \textbf{Les Protocoles Fondamentaux} (\emph{The Core Protocols}) version 3.03

\noindent Copyright \copyright{} Jim McCarthy et Michele McCarthy

\begin{small}
  \noindent \emph{The Core} est distribué selon les termes de la GNU General Public Licence telle que publiée par la Free Software
  Foundation, soit dans sa version 3 ou (à votre convenance) dans toute version ultérieure. Pour le contenu exact, voir
  \url{http://www.gnu.org/licenses/}. \emph{The Core} est considéré comme un code source en vertu de cet accord. Chacun est autorisé
  à copier et distribuer des copies conformes ou modifiées de ce contenu à condition que vous distribuiez également celui-ci
  dans son intégralité en incluant ce paragraphe.
\end{small}

Les engagements et les protocoles font partie intégrantes des Protocoles Fondamentaux.

\section{Engagements Fondamentaux}

\begin{enumerate}
	\item Lorsque je suis présent, je m'engage
	\begin{enumerate}
		\item à avoir conscience de et à communiquer
		\begin{enumerate}
			\item ce que je veux
			\item ce que je pense
			\item ce que je ressens
		\end{enumerate}
		\item à toujours chercher efficacement de l'aide
		\item à éviter et à refuser toute transmission émotionnelle incohérente
		\item dès l'émergence d'une idée plus pertinente que celle qui domine
		      à ce moment,
		\begin{enumerate}
			\item à la soumettre pour qu'elle soit acceptée ou refusée
			\item à explicitement chercher à l'améliorer
		\end{enumerate}
		\item à personnellement soutenir la meilleure idée
		\begin{enumerate}
			\item d'où qu'elle vienne
			\item quelle que soit mon espérance de voir une meil\-leure idée émerger
			\item quand je n'ai pas de meilleure alternative à proposer
		\end{enumerate}
	\end{enumerate}
	\item Je m'engage à chercher à percevoir plus que je ne cherche à être perçu
	\item Je m'engage à tirer parti des équipes, tout particulièrement pour venir à bout de tâches difficiles
	\item Je m'engage à parler quand et seulement quand je pense pouvoir améliorer le rapport \og{}résultat / efforts\fg{} courant
	\item Je m'engage à n'avoir et à n'accepter que des comportements et des échanges raisonnables, axés sur le résultat
	\item Je m'engage à quitter les situations moins constructives
	\begin{enumerate}
		\item quand je ne peux pas tenir les engagements demandés
		\item quand je peux prendre part à quelque chose de plus important
	\end{enumerate}
	\item Je m'engage à faire maintenant ce qui doit être fait et qui peut effectivement être fait maintenant
	\item Je m'engage à avancer vers un but donné et à adapter mon comportement pour l'atteindre
	\item Je m'engage à utiliser les Protocoles Fondamentaux (ou mieux) si possible
	\begin{enumerate}
		\item j'utiliserai de façon appropriée et sans porter préjudice les Contrôles de Protocole
	\end{enumerate}
	\item Je m'engage à ne jamais blesser --- ni à tolérer que quelqu'un blesse --- qui que ce soit pour sa fidélité à ces engagements
	\item Je m'engage à ne jamais rien faire d'idiot à dessein
\end{enumerate}


\end{document}
